% Этот шаблон документа разработан в 2014 году
% Данилом Фёдоровых (danil@fedorovykh.ru) 
% для использования в курсе 
% <<Документы и презентации в \LaTeX>>, записанном НИУ ВШЭ
% для Coursera.org: http://coursera.org/course/latex .
% Исходная версия шаблона --- 
% https://www.writelatex.com/coursera/latex/3.2

\documentclass[a4paper,12pt]{article}

%%% Работа с русским языком
\usepackage{cmap}					% поиск в PDF
\usepackage{mathtext} 				% русские буквы в фомулах
\usepackage[T2A]{fontenc}			% кодировка
\usepackage[utf8]{inputenc}			% кодировка исходного текста
\usepackage[english,russian]{babel}	% локализация и переносы

%%% Дополнительная работа с математикой
\usepackage{amsmath,amsfonts,amssymb,amsthm,mathtools} % AMS
\usepackage{icomma} % "Умная" запятая: $0,2$ --- число, $0, 2$ --- перечисление

%% Номера формул
%\mathtoolsset{showonlyrefs=true} % Показывать номера только у тех формул, на которые есть \eqref{} в тексте.
%\usepackage{leqno} % Немуреация формул слева

%% Свои команды
\DeclareMathOperator{\sgn}{\mathop{sgn}}

%% Перенос знаков в формулах (по Львовскому)
\newcommand*{\hm}[1]{#1\nobreak\discretionary{}
{\hbox{$\mathsurround=0pt #1$}}{}}

%%% Работа с картинками
\usepackage{graphicx}  % Для вставки рисунков
\graphicspath{{images/}{images2/}}  % папки с картинками
\setlength\fboxsep{3pt} % Отступ рамки \fbox{} от рисунка
\setlength\fboxrule{1pt} % Толщина линий рамки \fbox{}
\usepackage{wrapfig} % Обтекание рисунков текстом

%%% Работа с таблицами
\usepackage{array,tabularx,tabulary,booktabs} % Дополнительная работа с таблицами
\usepackage{longtable}  % Длинные таблицы
\usepackage{multirow} % Слияние строк в таблице

%%% Теоремы
\theoremstyle{plain} % Это стиль по умолчанию, его можно не переопределять.
\newtheorem{theorem}{Теорема}[section]
\newtheorem{proposition}[theorem]{Утверждение}
 
\theoremstyle{definition} % "Определение"
\newtheorem{corollary}{Следствие}[theorem]
\newtheorem{problem}{Задача}[section]
\newtheorem{definition}{Определение}
\newtheorem{example}{Пример}
 
\theoremstyle{remark} % "Примечание"
\newtheorem*{nonum}{Решение}

%%% Программирование
\usepackage{etoolbox} % логические операторы

%%% Страница
\usepackage{extsizes} % Возможность сделать 14-й шрифт
\usepackage{geometry} % Простой способ задавать поля
	\geometry{top=25mm}
	\geometry{bottom=35mm}
	\geometry{left=35mm}
	\geometry{right=20mm}
 %
\usepackage{fancyhdr} % Колонтитулы
 	\pagestyle{fancy}
 	\renewcommand{\headrulewidth}{0mm}  % Толщина линейки, отчеркивающей верхний колонтитул
 	%\lfoot{Нижний левый}
 	%\rfoot{Нижний правый}
 	%\rhead{Верхний правый}
 	%\chead{Верхний в центре}
 	%\lhead{Верхний левый}
 	% \cfoot{Нижний в центре} % По умолчанию здесь номер страницы

\usepackage{setspace} % Интерлиньяж
%\onehalfspacing % Интерлиньяж 1.5
%\doublespacing % Интерлиньяж 2
%\singlespacing % Интерлиньяж 1

\usepackage{lastpage} % Узнать, сколько всего страниц в документе.

\usepackage{soul} % Модификаторы начертания

\usepackage{hyperref}
\usepackage[usenames,dvipsnames,svgnames,table,rgb]{xcolor}
\hypersetup{				% Гиперссылки
    unicode=true,           % русские буквы в раздела PDF
    pdftitle={Заголовок},   % Заголовок
    pdfauthor={Автор},      % Автор
    pdfsubject={Тема},      % Тема
    pdfcreator={Создатель}, % Создатель
    pdfproducer={Производитель}, % Производитель
    pdfkeywords={keyword1} {key2} {key3}, % Ключевые слова
    colorlinks=true,       	% false: ссылки в рамках; true: цветные ссылки
    linkcolor=red,          % внутренние ссылки
    citecolor=green,        % на библиографию
    filecolor=magenta,      % на файлы
    urlcolor=blue           % на URL
}

%\renewcommand{\familydefault}{\sfdefault} % Начертание шрифта

\usepackage{multicol} % Несколько колонок

\usepackage{listings}

\definecolor{mygreen}{rgb}{0,0.6,0}
\definecolor{mygray}{rgb}{0.5,0.5,0.5}
\definecolor{mymauve}{rgb}{0.58,0,0.82}

\lstset{ %
	backgroundcolor=\color{white},   % choose the background color; you must add \usepackage{color} or \usepackage{xcolor}
	basicstyle=\footnotesize,        % the size of the fonts that are used for the code
	breakatwhitespace=false,         % sets if automatic breaks should only happen at whitespace
	breaklines=true,                 % sets automatic line breaking
	captionpos=b,                    % sets the caption-position to bottom
	commentstyle=\color{mygreen},    % comment style
	deletekeywords={...},            % if you want to delete keywords from the given language
	escapeinside={\%*}{*)},          % if you want to add LaTeX within your code
	extendedchars=true,              % lets you use non-ASCII characters; for 8-bits encodings only, does not work with UTF-8
	frame=single,                    % adds a frame around the code
	keepspaces=true,                 % keeps spaces in text, useful for keeping indentation of code (possibly needs columns=flexible)
	keywordstyle=\color{blue},       % keyword style
	language=Octave,                 % the language of the code
	morekeywords={*,...},            % if you want to add more keywords to the set
	numbers=left,                    % where to put the line-numbers; possible values are (none, left, right)
	numbersep=5pt,                   % how far the line-numbers are from the code
	numberstyle=\tiny\color{mygray}, % the style that is used for the line-numbers
	rulecolor=\color{black},         % if not set, the frame-color may be changed on line-breaks within not-black text (e.g. comments (green here))
	showspaces=false,                % show spaces everywhere adding particular underscores; it overrides 'showstringspaces'
	showstringspaces=false,          % underline spaces within strings only
	showtabs=false,                  % show tabs within strings adding particular underscores
	stepnumber=1,                    % the step between two line-numbers. If it's 1, each line will be numbered
	stringstyle=\color{mymauve},     % string literal style
	tabsize=2,                       % sets default tabsize to 2 spaces
	title=\lstname                   % show the filename of files included with \lstinputlisting; also try caption instead of title
}

\usepackage{qtree}

\author{Селегей В.П.}
\title{Введение в компьютерную и корпусную лингвистику}
\date{\today}

\begin{document}
	
	\maketitle
	
	\setcounter{section}{-1}
	
	{\flushright \it 	
		Это всё будет обновляться и дизайниться\\ по мере хода лекций (надеюсь) и по мере отсутствия лени у меня (верю).
		
		Буду рад обсудить разные фишечки live-TeXинга с теми, кто практикует.
		
		AQ %Куликов Алексей, гр. 397
		
	%	\url{https://github.com/alexeyqu/mipt-alexeyqu}
	}
	
	\section { Предисловие }
	
	Все предметы на КЛ делятся на 4 блока:
	
	\begin{itemize}
		
		\item База Физтеха: теория графов, автоматов, формальные грамматики.
		
		\item Лингвистика: сначала введение (в прошлые года в бакалавриате не было), в магистратуре будет уже по темам: морфология, синтаксис, семантика.
		
		\item Обзор моделей для лингвистики: различные open-source прикладные проекты
		
		\item Задачи: машинный перевод, распознавание и т.д.
			
	\end{itemize}
	
	Вот здесь будет второй пункт -- базовые понятия лингвистики.
	
	Первые 2 лекции будут совсем вводные в лингвистику и компьютерную лингвистику, дальше уже по сути пойдём.
	
	\section { Лекция 1. 18.02.2016 }
	
	\subsection { Чем занимается лингвистика? }
	
	Итак, чем же занимается лингвистика?
	
	Обычно под "лингвистом" подразумевается человек, знающий языки.	Здесь определяем более узкий срез понятия: 
	
	\begin{definition}
		 Лингвистика теоретическая -- наука о естественном человеческом языке вообще и о всех языках мира как индивидуальных его представителях.
	\end{definition}
	
	Сейчас современная наука (напр. на конференциях) более тяготеет к разнообразию в исследуемых языках, как следствие, необходимо развивать общие формальные модели.
	
	Всё это ведёт к методологическому переходу от субъективной интроспекции (самоосознание внутри одного языка) к объективным корпусным и нейролингвистическим методам исследований.
	
	3 кита теоретической лингвистики: типологический подход, формальные представления, объективные связи. % TODO здесь мб дополнить из презы
	
	Прикладная же лингвистика занимается NLP (Natural Language Processing).
	
	\subsection { А "компьютерная" лингвистика (КоЛинг)?}
	
	Суть КоЛинг -- автоматическая обработка ЕЯ (естественного языка) в научных или прикланых целях.
	
	КоЛинг как часть лингвистики:
	
	\begin{itemize}
		\item Теория: создание формальных моделей языка ({\it аналогия: модель расширяющейся Вселенной})
	
		\item Практика: применение компьютеров в лингвистических исследованиях ({\it аналогия: компьютерная медицина})
	
		\item Инженерная деятельность (наша проекция практики): решения с помощью компов ({\it парсеры, корпуса, лингворесурсы}) различных задач обработки ЕЯ ({\it маш. перевод звучащей речи, обработка запросов на ЕЯ, анализ новостей, генерация репортажа по видео, голосовое управление аппаратом и т.д.})
	\end{itemize}
	
	
	В результате получается эдакий "Лингвистический пылесос".
	
	В идеале между Линг и КоЛинг будет связь: модели языка от гуманитариев соединяются с методами получения языковых данных от прогеров ({\it или наоборот, я что-то запутался}), что ведёт к общей пользе ура.

	\subsection { Окей, как это смоделировать? }
	
	\begin{definition}
		Язык -- это устройство для кодирования значений с помощью системы специальных средств в целях коммуникации.
	\end{definition}
	
	А что такое "значение"? Мысль? Идея? Её нельзя положить на стол, препарировать и распознать.
	
	Есть различные структуры-попытки формализовать всё это.
	
	// Теория Роджера Шенка (wiki) % TODO добавить про неё
	
	// Compreno-модель % TODO мб после проекта будет понятнее, добавить
	
	В принципе, можно пытаться снять энцефолограмму мозга при различных мыслях, но по понятным причинам это слишком сложно, чтобы работать. Пока что люди до анализа таких данных не доросли.
	
	\subsection { Естественный язык: basics }
	
	Основоположником всего этого был, конечно, Ноам Хомский (Chomsky).
	
	Итак, язык -- это кодирующая система.
	
	Якобсон пытался формализовать весь процесс коммуникации (адресант, адресат, код, котнакт, сообщение и т.д.) % TODO добавить про Якобсона
	
	Само кодирование смысла получается через различные средства: на самом верхнем уровне они делятся на лексические и грамматические средства.
	
	Далее, если высказыание на ЕЯ -- это фраза, предложение, то описывать (иными словами, пытаться формализовать) их мы можем на поверхностном (например, в виде синтаксического анализа, довольно легко) и глубинном (что-то типа семантического анализа, гораздо сложнее) уровнях.
	
	\begin{example}
		Мы можем сказать, что "в аудитории 16 человек" великим множеством способов ==> язык как средство выражения чудовищно избыточен.
	\end{example}
	
	Кроме того, надо понять, что смысл передаётся не только языковыми средствами (мимика, жесты, интонирование). Все их надо учитывать.
	
	\subsection { Средства описания значения }
	
	Как мы можем представлять фразу математически? (простейший случай -- bag of words)
	
	Из проги можем вспомнить деревья-графы, мб даже семантические сети.
	
	В таком случае, какая глубина деревьев и вообще как их настраивать?
	
	В итоге:
	
	\begin{itemize}	% TODO мб подправить, как-то стрёмно это выглядит
		\item глубина описаний
	
		\item язык описания значений
	
		\item связь с неязыковыми структурами.
	\end{itemize}
	
	\subsection { Грамматические средства кодирования }
	
	Попробуем перечислить все уровни выражения мыслей. Начинаем с кирпичиков-слов.
	
	Идеи: Оттуда берутся словоформы, куда затем можно добавить служебные слова. Потом организуется связь с артикуляцией через пунктуацию.
	
	Систематизируем: 
	
	\begin{itemize} % TODO мб подправить примеры
		\item[1] Самое главное и образующее свойство - линейный порядок слов. "пять человек" и "человек пять" - разница в примерности, которую даёт лишь порядок слов
		
		Ещё: "Mary gave John an apple" vs "John gave Mary an apple".
		
		\item[2] Словоформы и словообразование.
		
		"Отец купил сыну..." и "Отцу купил сын"
		
		"прыгнуть-подпрыгнуть-перепрыгнуть"
		
		"выпить -- выпивать"
		
		\item[3] Вспомогательные слова
		
		\item[4] Пунктуация
		
		\item[5] Просодика: акценты, паузы, интонация
		
		\item[6] Новые средства кодирования: смайлы, зачёркивания, гипертекст, шрифт и прочая графика.
	\end{itemize}
	
	\subsection{ Простые примеры }
	
	Предложение: 
	
	{\it Этот парень делает любую вещь, к которой прикоснётся, неисправной}.
	
	Вообще, школьный синтаксический разбор суть обыкновенное дерево разбора. А что тогда корень дерева? Подлежащее или сказуемое? В школе думают о первом, но это чушь.
	
	Главное слово должно объяснять много о устройстве предложения. В данном случае "парень" не даёт почти никакой информации, это может быть кто(что) угодно без ущерба смыслу. Значит, лучше думать о глаголе.
	
	Итак, "делает" - вершина дерева.
	
	Делает -- кто? Парень (именительный падеж) А детектим его через число+падеж (изменяемые, берём окончание --> зависит от конкретного предложения и определяет предложение)+род+одушевлённость (классифицирующие: отсюда берём и смысл)
	
	Вещь -- определяем падеж по наличию зависимости "любую" (неодуш, ...).
	
	Неисправной: падеж берётся от глагола, а всё остальное -- от "вещи". ещё одно такое слово -- "(к) которой", у них по 2 родителя де-факто. Т.е. чисто древесная структура не работает.
	
	Де-факто тут есть эллипсис. Это когда мы неполно опускаем смысловые слова (мб повторяющиеся и т.д.)

	\subsection { Система уровней анализа языка }
	
	\begin{itemize}
		\item[0] Лексический анализ (слова, знаки препинания, цифры и т.д.)
		
		\item[1] Морфологический анализ (все возможные грамматические характеристики выделенных лексем)
		
		\item[2] Синтаксический анализ (установление базовых связей между словами)
		
		\item[] {\it Тут начался интерпретационный уровень}
		
		\item[3] Семантический анализ (понятие структуры, связанной со смыслом, но при этом мы остаёмся в границах языка)
		
		\item[4] Прагматический анализ (интерпретация семантики в контексте конкретной ситуации, отнологии (общих знаний об устройстве мира))
	\end{itemize}
	
	картинка про треугольник перевода от Селегея
	
	\subsection { Задачи и уровни их решения } % TODO CONTINUE
	
	1) Проверка орфографии
	
	2) Машинный переносы

\end{document}